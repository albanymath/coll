% !TeX program = xelatex
\documentclass[colloquium]{UAmathtalk}

\author{Kush R. Varshney}
\address{Thomas J. Watson Research Center, IBM}
\urladdr{https://researcher.watson.ibm.com/researcher/view.php?person=us-krvarshn}
\title{On the Intersection of\\ Machine Learning and Society}
\date{Friday, October 14, 2022}
\renewcommand*{\where}{ES-241}
\renewcommand*{\moreinfo}{(tea \&\ coffee at 2:30 p.m.)}%


\begin{document}

\maketitle

\begin{abstract}
Machine learning technologies are growing ever more powerful every minute, including through work we’ve conducted at IBM Research. They now continually interact with society in our daily lives, livelihoods, and liberties. Some machine learning researchers are just happy to see and contribute to the progress, while others are afraid of the possible societal injustices that may be wrought and work to prevent a dystopian future. In this talk, I will lay out the landscape of the intersection of machine learning and society, focusing on four aspects: social good, ethics principles, trustworthy techniques, and critical theory. I will further report on which parts of this research landscape different researchers tend to contribute most to: purely technical vs. sociotechnical. I will also hypothesize possible underlying motivations for these researchers.
\end{abstract}

\vfill\noindent\textsc{\large This colloquium talk is presented jointly by the Department of Mathematics and Statistics and the University at Albany Asian Coalition of Professionals (UA-ACP)}

\end{document}
