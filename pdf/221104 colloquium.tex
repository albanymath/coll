% !TEX TS-program = XeLaTeX
\documentclass[colloquium]{UAmathtalk}

\author{Karin Reinhold--Larsson}
\address{University at Albany, SUNY}
\urladdr{https://www.albany.edu/~reinhold/}
\title{The Ritt Property for Contraction Operators and Square Functions in~$L^1$}
\date{Friday, November 4, 2022}
\renewcommand*{\where}{BB-B012}
\renewcommand*{\moreinfo}{(tea \&\ coffee at 2:45 p.m.)}%


\begin{document}

\maketitle

\begin{abstract}
The problem of convergence and rate of convergence of powers of contraction operators $T^n$  has fascinated mathematicians. 
In recent years, advancements  for operators satisfying the following Ritt condition: \[\sup_n n \|T^n-T^{n+1}\|<\infty,\] 
 were obtained using a Cauchy integral representation for operators for functions in $L^p$, 
 for $1<p<\infty$.   We will present the history of the problem, with spectral requirements for pointwise convergence and the boundedness of a square functions such as  $\sum_n n |T^nf-T^{n+1}f|^2$. 
Unfortunately, the methods did not extend to $L^1$. However, positive results are obtained for the following particular case:
let  $\mu$ be a probability measure in the integers, and  $\tau:X \to X$  a measure preserving transformation. 
Define \[T_{\mu} (f)(x)= \sum_{k=-\infty}^{\infty} \mu(k) f(\tau^k x).\] 
We'll discuss necessary conditions on the measure $\mu$ that yield the Ritt property and convergence of certain squares functions in $L^1$.
\end{abstract}

\end{document}
