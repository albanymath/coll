% !TeX program = xelatex
\documentclass[colloquium]{UAmathtalk}

\author{Laurent Saloff-Coste}
\address{Cornell University}
\urladdr{https://math.cornell.edu/laurent-saloff-coste}
\title{The Multiplayer Gambler’s Ruin Problem}
\date{Friday, October 21, 2022}
\renewcommand*{\where}{BB-B010}
\renewcommand*{\moreinfo}{(tea \&\ coffee at 2:45 p.m.)}%


\begin{document}

\maketitle

\begin{abstract}
Two players repeatedly play a fair game, exchanging one token after each game while the total number of tokens in play, $N$, is fixed. If player~A starts with $x$ tokens, what is the probability that they will end up with all the tokens?  This is a basic question for undergraduate probability course and the answer is~$x/N$. In this talk, I will discuss the gambler’s ruin problem involving three or more players.  Suppose three players play as follows: a total of~$N$ tokens are in play and $(a,b,c)$ is the initial distribution of tokens between the three players. At each stage, a random pair of the three players plays a fair game and exchange one token. Let~$P(N,A,a,b,c)$ be the probability that player~A ends up loosing all their tokens first. $P(N,A,N-2,1,1)$ is the probability that the very dominant player A starting with most the tokens ends up loosing first. Can you guess the order of magnitude of~$P(N,A,N-2,1,1)$?
\end{abstract}

\end{document}
